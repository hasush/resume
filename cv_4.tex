%%%%%%%%%%%%%%%%%%%%%%%%%%%%%%%%%%%%%%%%%
% Medium Length Professional CV
% LaTeX Template
% Version 2.0 (8/5/13)
%
% This template has been downloaded from:
% http://www.LaTeXTemplates.com
%
% Original author:
% Trey Hunner (http://www.treyhunner.com/)
%
% Important note:
% This template requires the resume.cls file to be in the same directory as the
% .tex file. The resume.cls file provides the resume style used for structuring the
% document.
%
%%%%%%%%%%%%%%%%%%%%%%%%%%%%%%%%%%%%%%%%%

%----------------------------------------------------------------------------------------
%	PACKAGES AND OTHER DOCUMENT CONFIGURATIONS
%----------------------------------------------------------------------------------------

\documentclass{resume} % Use the custom resume.cls style

\usepackage[left=0.75in,top=0.6in,right=0.75in,bottom=0.6in]{geometry} % Document margins
\usepackage{longtable}

\name{Gursharan Yash Singh Sandhu} % Your name
%\address{15 East Kirby St. Apt 1017 \\ Detroit, MI 48202} % Your address
\address{115 North Wilson Drive \\ Royal Oak, MI 48067} % Your address
%\address{123 Pleasant Lane \\ City, State 12345} % Your secondary addess (optional)
\address{(734)~$\cdot$~255~$\cdot$~6775 \\ gys.sandhu@gmail.com} % Your phone number and email
%\address{Detroit, MI \\ gys.sandhu@gmail.com}

\begin{document}

%----------------------------------------------------------------------------------------
%	EDUCATION SECTION
%----------------------------------------------------------------------------------------

\begin{rSection}{Education}

\textbf{Udacity Deep Learning Nano Degree}\ Atlanta, GA \hfill {\em May 2018}
\bigskip \\
\textbf{Georgia Institute of Technology}\ Atlanta, GA \hfill {\em Winter 2019} \\ 
Masters in Computer Science: Machine Learning Specialization \smallskip \\
Overall GPA: 4.0/4.0
\bigskip \\
\textbf{Wayne State University}\ Detroit, MI \hfill {\em August 2015} \\ 
Ph.D. in Physics \smallskip \\
Overall GPA: 3.74/4.0
\bigskip \\
\textbf{University of Michigan}\ Ann Arbor, MI\hfill {\em May 2010} \\ 
B.S. in Physics \& Mathematics (Dual Degree)\smallskip \\
%Minor in Linguistics \smallskip \\
Mathematics and Physics GPA: $\pi/4$%3.14

\end{rSection}

%----------------------------------------------------------------------------------------
%	WORK EXPERIENCE SECTION
%----------------------------------------------------------------------------------------

\begin{rSection}{Experience}

\begin{rSubsection}{Delphinus Medical Technologies, Inc.}{August 2014 - Present}{Imaging Scientist}{Novi, MI}
\item Key leader in the weekly planning and design of company strategy for the development of the Delphinus ultrasound tomography device.
\item Lead research scientist and software developer for commercial grade machine learning software for application in the detection of cancer in a clinical setting.
\item Managed large clinical team for data set generation and breast tumor library creation. This involved working with clinical coordinators and research personnel to identify and recruit patients and then reconstruct their images. The resultant images were then aggregated into our large clinical data base.
\item Developed algorithms that perform automatic segmentation and detection of different types of tissues in ultrasound tomography images using unsupervised machine learning techniques.
\item Developed texture and other feature extraction methods for ultrasound tomography images. Used the resulting features in combination with supervised machine learning methods such as neural networks and support vector machines to classify and predict the presence of cancer in ultrasound tomography images.
\item Developed techniques that use neural network supervised learning techniques to automatically adjust the brightness and contrast of medical images based on optimal settings obtained from manual selection by a radiologist.
\item Mentored and oversaw the research of interns including investigations into stiffness imaging fusion modalities and 3D image viewing tools.
\item Investigated the use of novel cost functions and log-scale domain methods for image reconstruction inverse problems.
\item Developed post-processing image reconstruction software that visualizes and rotates 3D volumes of interest.
\item Developed image fusion algorithms which create new imaging modalities from ones that are already existing.
\item Performed many different forms of post-processing in order to enhance medical images.
\item Developed 3D waveform tomography algorithms which model the acoustic wave equation in 3D. The resulting fields are then back propagated with a gradient descent algorithm to solve acoustic inverse problems.
\item Reconstructed the acoustic parameter of density using numerical simulations.
\item Performed source encoding work where image reconstruction times can be improved by firing multiple transducer elements in parallel and then decoding their resultant wave fields.
\item Modeled ultrasound transducer array beam patterns for improved image reconstruction by using variable phases and delays applied to finite difference meshes. This involved using advanced interpolation techniques to properly weigh grid indices to match proper Cartesian coordinates.  
\item Wrote fast sparse matrix solvers using in-house built algorithms.
\item Performed CPU and GPU software optimization to accelerate image reconstruction algorithms to clinically accepted time scales.
\item Developed 2D waveform tomography algorithms to quantify the sound speed and attenuation properties of the breast using gradient descent, finite difference, and matrix inversion methods of the Helmholtz equation.
\item Designed and developed image viewing software that allows a radiologist to read medical images. This includes tools to help diagnostic capabilities and work flow of reading medical images.
\item Designed ultrasound transducer arrays. In particular, developed revolutionary new state-of-the-art low frequency ultrasound tomography transducer arrays. This includes transducer hardware design and parameter selection using specifications of the response and beam patterns using simulations.
\item Contributed to the development of embedded software written in Python on the Delphinus ultrasound tomography device. The system consists of a large collection of event driven signaling mechanisms.
\item Contributed to radiology methods in the assessment of the presence of cancer within ultrasound tomography images.
\item Contributed in different ways to the software architecture of the Delphinus ultrasound tomography device. This ranges from simple helper and get set functions in the C++ architecture to very mathematically rigorous image reconstruction algorithms.
\item Applied various signal and image processing algorithms and techniques to preprocess ultrasound signals and post process ultrasound tomography images.

%\item Developed texture and other feature extraction methods for ultrasound tomography images. Used the resulting features in combination with supervised machine learning methods such as neural networks and support vector machines to classify and predict the presence of cancer in ultrasound tomography images.
%\item Developed 3D waveform tomography algorithms using similar but more complicated approaches to the 2D versions.
%\item Developed 2D waveform tomography algorithms to quantify the sound speed and attenuation properties of the breast using gradient descent, finite difference, and matrix inversion methods of the Helmholtz equation.
%\item Contributed to the design and development of image viewing software that allows a radiologist to read medical images. This includes tools to help the diagnostic capabilities and work flow of reading medical images.
%\item Contributed to the design and development of methods to model the physical beam response of ultrasound transducer array.
%\item Contributed to the design and development of low-frequency ultrasound transducer arrays.
%\item Contributed to the development of embedded software on the Delphinus ultrasound tomography device.
%\item Contributed to the development of methods in radiology to assess the presence of cancer within ultrasound tomography images.
%\item Contributed in different ways to the software architecture of the Delphinus ultrasound tomography device. This ranges from simple helper and get set functions in the C++ architecture to very mathematically rigorous image reconstruction algorithms.
%\item Applied various signal and image processing algorithms and techniques to preprocess ultrasound signals and post process ultrasound tomography images.
%\item Was a key leader in the weekly planning and design of company strategy for the development of the Delphinus ultrasound tomography device.
\end{rSubsection}

%------------------------------------------------

\begin{rSubsection}{Wayne State University}{August 2012 - August 2014}{Graduate Research Assistant}{Detroit, MI}
\item Performed frequency domain waveform breast sound speed and attenuation tomography by translating and adopting work in Geophysics and applying them to medical ultrasound  transducer arrays.
\item Analyzed and assessed the acoustic properties of human breast tissue by the use of ultrasound tomography.
\end{rSubsection}

%------------------------------------------------

\begin{rSubsection}{Wayne State University}{August 2010 - August 2012}{Graduate Teaching Assistant}{Detroit, MI}
\item Instructor for Physics laboratory class and Physics learning center. 
\item Conducted genomic research using unsupervised machine learning techniques such as PCA and clustering in mapping biological markers of breast cancer using Raman spectroscopy.
\item Researched Monte Carlo modeling of electron energy deposition stochastic processes in radiological dose calculations.
\item Wrote high-frequency wave equation approximation algorithms which solved the inverse problem of assessing the sound speed and attenuation properties of mediums. Successfully applied these techniques to reconstruct the acoustic properties of sound speed and attenuation of the human breast using data gathered by ultrasound tomography transducer arrays.
\end{rSubsection}

\end{rSection}

%----------------------------------------------------------------------------------------
%	TECHNICAL STRENGTHS SECTION
%----------------------------------------------------------------------------------------
\newpage
\begin{rSection}{Technical Strengths}

%\begin{tabular}{ @{} >{\textbfseries}l @{\hspace{6ex}} l }
%\textbf{Computer Languages} \quad C/C++, \LaTeX, CUDA, Python, Java, Fortran, bash \\
%\textbf{Tools} \quad SVN, XML, Vim, Matlab, ImageJ, Mathematica, SPSS, Linux, Unix, Eclipse \\
%\textbf{Computational Skills} \quad signal processing, image processing, numerical modeling, LU factorization, solutions to sparse linear systems, gradient descent methods, medical devices, finite difference method, frequency and time domain waveform simulation, statistical analysis, machine learning, GPU, image analysis, inverse problem, software optimization, numerical optimization \& modeling, aray data processing\\
%\textbf{Medical Physics} \quad dosimetry, radiation therapy, imaging, radiation safety, image quality assurance, treatment planning, nuclear medicine, ultrasound, CT, MRI, tomography, breast imaging, diagnostic radiology,
%\end{tabular}
\begin{tabular}{p{5cm}p{11cm}}
\textbf{Computer Languages} &  C/C++, CUDA, Fortran, Java, \LaTeX, and Python\\
\textbf{Tools} &  Git, IDEs (Eclipse, Netbeans, PyCharm), ImageJ, Keras, Linux, Mathematica, Matlab, Numpy, Pandas, TensorFlow, SciPy, Sci-kit Learn, SPSS, SVN, Varian Eclipse, and XML \\
\textbf{Skills} & aray data processing, artificial intelligence, data mining, deep learning (CNN, RNN, GAN), finite difference method, feature extraction, frequency and time domain waveform simulation, GPU, gradient descent methods, image analysis, image processing, inverse problem, LU factorization, machine learning, medical devices, numerical optimization \& modeling, reinforcement learning, signal processing, statistical analysis, software optimization, and solutions to sparse linear systems\\
\textbf{Medical Physics} & breast imaging, CT, diagnostic radiology, dosimetry, image quality assurance, imaging, MRI, nuclear medicine, radiation safety, radiation therapy, radiology, tomography, treatment planning, and ultrasound
\end{tabular} 
%,   numerical methods,   biomedical engineering,
\end{rSection}
\
%----------------------------------------------------------------------------------------
%	Publications and Patents
%----------------------------------------------------------------------------------------

\begin{rSection}{Publications and Patents}
\item Sandhu, G., Littrup, P., Sak, M., Li, C., \& Duric, N. (2018, March). Ultrasound tomography supervised machine learning. In SPIE Medical Imaging. International Society for Optics and Photonics.
\item Sandhu, G., Littrup, P., \& Marooghy, M. (2017, December). Machine Learning and Data Mining on Ultrasound Tomography Images Using a Region-of-Interest Tool. U.S. Patent Pending. Washington, DC: U.S. Patent and Trademark Office.
\item Duric, N., Littrup, P., Sandhu, G., Krycia, M., \& Sak, M. (2017, November). Waveform Enhanced Reflection and Margin Boundary Characterization For Ultrasound Tomography. U.S. Patent Pending. Washington, DC: U.S. Patent and Trademark Office.
\item Sandhu, G., Littrup, P., Sak, M., Li, C., \& Duric, N. (2017, November). Tissue characterization with ultrasound tomography machine learning. In Medical Ultrasound Tomography.
\item Li, C., Sandhu, G.S., Boone, M., Duric, N., Littrup, P., Sak, M., \& Bergman, K. (2017, November). Breast tissue characterization with sound speed and stiffness imaging. In Medical Ultrasound Tomography.
\item Duric, N., Littrup, P., Li, C., Sak, M., Sandhu, G., Bergman, K., Boone, M., \& Chen, Di. (2017, November). Ultrasound tomography for breast cancer screening. In Medical Ultrasound Tomography.
\item Littrup, P.J., Duric, N., Li, C., Sak, M., Sandhu, G., Bergman, K., Boone, M., \& Chen, Di. (2017, November). Current challenges in breast screening and diagnosis: From molecules to peritumoral regions and radiomics – The emerging imaging of whole breast stiffness. In Medical Ultrasound Tomography.
\item Sandhu, G. S., Duric, N., Li, C., Roy, O., \& West, E. (2017). Tissue imaging and analysis using ultrasound waveform tomography. U.S. Patent No. WO2017040866A1. Washington, DC: U.S. Patent and Trademark Office.
\item Sandhu, G. Y., West, E., Li, C., Roy, O., \& Duric, N. (2017, March). 3D frequency-domain ultrasound waveform tomography breast imaging. In SPIE Medical Imaging. International Society for Optics and Photonics.
\item Li, C., Sandhu, G. Y., Boone, M., \& Duric, N. (2017, March). Breast Imaging Using Waveform Attenuation Tomography. In SPIE Medical Imaging (pp. 101390A-101390A). International Society for Optics and Photonics.
\item Sandhu, G. S., Li, C., Roy, O., Schmidt, S., \& Duric, N. (2016). Ultrasound waveform tomography method and system. U.S. Patent No. US20160030000A1. Washington, DC: U.S. Patent and Trademark Office.
\item Sandhu, G. Y. S., Li, C., Roy, O., West, E., Montgomery, K., Boone, M., \& Duric, N. (2016, April). Frequency-domain ultrasound waveform tomography breast attenuation imaging. In SPIE Medical Imaging (pp. 97900C-97900C). International Society for Optics and Photonics.
\item Sandhu, G. Y., Li, C., Roy, O., Schmidt, S., \& Duric, N. (2015). Frequency domain ultrasound waveform tomography: breast imaging using a ring transducer. Physics in medicine and biology, 60(14), 5381.
\item Sandhu, G. Y., Li, C., Roy, O., Schmidt, S., \& Duric, N. (2015, March). High-resolution quantitative whole-breast ultrasound: in vivo application using frequency-domain waveform tomography. In SPIE Medical Imaging (pp. 94190D-94190D). International Society for Optics and Photonics.
\item Li, C., Sandhu, G. S., Roy, O., Duric, N., Allada, V., \& Schmidt, S. (2014, March). Toward a practical ultrasound waveform tomography algorithm for improving breast imaging. In SPIE Medical Imaging (pp. 90401P-90401P). International Society for Optics and Photonics.
\end{rSection}

%----------------------------------------------------------------------------------------

%----------------------------------------------------------------------------------------
%	Posters, Presentations, and Workshops
%----------------------------------------------------------------------------------------

\begin{rSection}{Posters, Presentations, and Workshops}
\item Ultrasound tomography supervised machine learning. (2018, February). Presentation at SPIE Medical Imaging, Houston, Texas.
\item The importance of peritumoral comparisons by ultrasound tomography: Radiomics and breast mass discrimination. (2017, November). Presentation at Radiological Society of North America, Chicago, Illinois.
\item Tissue characterization with ultrasound tomography machine learning. (2017, November). Presentation at Medical Ultrasound Tomography Conference, Speyer, Germany.
\item 3D Frequency-Domain Ultrasound Waveform Tomography Breast Imaging. (2017, February). Presentation at SPIE Medical Imaging, Orlando, Flordia.
\item Ultrasound Tomography Breast Cancer Imaging. (2017, January). Presentation at Life After Grad School Seminar at University of Michigan Department of Physics, Ann Arbor, Michigan.
\item Toward High Resolution Whole Breast Imaging using Ultrasound Tomography: A Comparison with MRI. (2016, November). Presentation at Radiological Society of North America, Chicago, Illinois.
\item Frequency-Domain Ultrasound Waveform Tomography Breast Attenuation Imaging. (2016, February). Presentation at SPIE Medical Imaging, San Diego, California..
\item High-resolution quantitative whole-breast ultrasound: in vivo application using frequency-domain waveform tomography. (2015, March). Presentation at SPIE Medical Imaging, Orlando, Florida.
\item Breast Imaging With Ultrasound Tomography. (2014, November). Presentation at Physics 464: Special Techniques in Health Physics, University of Windsor, Windsor, Canada.
\item Breast Imaging With Ultrasound Waveform Tomography. (2014, October). Presentation at Condensed Matter Seminar, Wayne State University, Detroit, Michigan.
\item GPU Programming for Medical Physics and Medical Imaging Research. (2014, October). Workshop at Department of Radiation Oncology, University of Texas Southwestern Medical Center, Dallas, Texas.  
\item Toward a practical ultrasound waveform tomography algorithm for improving breast imaging. (2014, March).  Poster at SPIE Medical Imaging, San Diego, California.
\item Tomographic Reconstruction of Breast Characteristics Using Transmitted Ultrasound Signals. (2012, October). Poster in APS Ohio Sections Fall Meeting Abstracts (Vol. 1), Detroit, Michigan.

\end{rSection}

%----------------------------------------------------------------------------------------
%	Personal Information
%----------------------------------------------------------------------------------------

\begin{rSection}{Personal Information}
\item Languages: English, Punjabi, Hindi, and Spanish (basic)
\item U.S. Citizen
\item Professional percussionist: Western classical, rudimental, drum set, North \& South Indian classical, Arabic/Turkish/Persian classical, Afro-Cuban, Indonesian gamelan, rocks, blues, jazz, funk, hip-hop, fusion
\end{rSection}

%----------------------------------------------------------------------------------------


%----------------------------------------------------------------------------------------


%----------------------------------------------------------------------------------------
%	EXAMPLE SECTION
%----------------------------------------------------------------------------------------

%\begin{rSection}{Section Name}

%Section content\ldots

%\end{rSection}

%----------------------------------------------------------------------------------------

\end{document}
